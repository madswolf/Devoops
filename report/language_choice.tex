\subsubsection{Our measures}
One of the hardest decisions to begin this project, was to pick the programming language that would work the best. A couple of things we had to take into account were performance, implementation time, difficulty of usage, support, stability, maturity, feature mapping and many more. 
\subsubsection{Python}
Initially, we considered working with Python and Django, since that would make refactoring the initial application much easier while also making the development time extremely fast due to the structure of the language, and the group's experience with Python. After careful consideration, we decided to give up on this idea, since Python does not support explicit typing, which might result in some unnoticed runtime errors. The other factor that influenced our decision was the "running" speed. Python is slower than other considered languages, and it does not support multithreading natively, so we would need to use special libraries, which would result in a higher memory overhead and would detract from the ease of use which python in known for.
\subsubsection{Functional Languages}
We considered functional languages as well, however we gave up on the idea rather fast, because most of our team is not as experienced in functional paradigm, meaning the development of the application would be more difficult. A major downside was also our lack of knowledge or experience of the frameworks present in these ecosystems.
\subsubsection{C\#}
After some consultation, we decided to go for a language that is object-oriented, since we all know this paradigm quite well. We also wanted to pick a language that has a fast execution time. Based on these criteria, we decided to settle for C\#, since it is relatively fast compared to Python for instance. It also has a lot of support in terms of online documentation and libraries, and is really mature in terms of feature it supports. C\# is also the choice for many in the corporate world, which also was a considerable benefit.
This choice is not necessarily the definitive best choice, as every programming language has their respective pros and cons, but we decided that C\# was most fitting choice for our group, and allowed us to expand our knowledge in other areas, such as Razor Pages as well as the new technologies.