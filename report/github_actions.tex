The need for a good CI/CD pipeline required us to find a tool, that could easily integrate into our workflow. The simplest solution, was to use the built-in solution provided by GitHub: GitHub Actions, which allows one to create pipelines by using \textit{.yaml} files, and then to inspect their results on GitHub. One can argue that GitHub Actions is a poor choice, since we would rely even more on GitHub being online, and if GitHub goes down, we would lose access to all the pipelines. For that reason, we were considering using Travis CI, however we did not want to pay out of pocket for a tool for the project, so we settled on GitHub Actions, and accepted the risk of the service being offline.
Even when using other CI/CD solutions, they would have to be integrated into our repository, in order to allow for functionalities like testing on pushes to a branch. GitHub would have to be online for that to work anyway.

\subsubsection{Automatic releases}
As part of the project we had to do regular releases of our code. GitHub Actions can automatically release when new code is pushed to master, which we have set up, such that the latest release always has the newest version without us having to write anything.